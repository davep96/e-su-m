\documentclass[a4paper,11pt]{article}

\usepackage[T1]{fontenc}

\usepackage[utf8]{inputenc}

\usepackage[italian]{babel}

\usepackage{graphicx}

\usepackage{indentfirst}

\usepackage{amsmath,amssymb}

\usepackage{enumitem} 

\newcommand{\virgolette}[1]{``#1''}

\usepackage[margin=1in]{geometry} %Smaller margins

\usepackage{lmodern} %Vector PDF

\usepackage{siunitx}

\usepackage{xcolor}

\usepackage{colortbl}

\usepackage{booktabs}

\usepackage{graphicx}
\graphicspath{ {../Immagini/} }

\usepackage{wrapfig}

\usepackage{siunitx} % Per unit� di misura in generale e la corretta rappresentazione dei numeri.


\begin{document}
	
	Per una corretta visualizzazione dei numeri, con le unità di misura e gli ordini di grandezza usare il pacchetto siunitx con i comandi:\\
	
		\SI{5}{\centi\meter}
	\\
		\SI{633e-9}{\meter}
	\\
		\num{6.022e23}
	\\
	Inoltre usiamo la convenzione internazionale per cui i decimali si separano con un punto per piacere.
	
	equazioni e labels:
	\begin{equation}\label{regressione}  %relazione lineare per m/e
	(B_z R)^2 = 2\Delta V \, m/e 
	\end{equation}
	
	\begin{equation}\label{B_terra}    %campo magnetico terra
	B_t= \frac{I}{I_0} (B_z \cot \theta + B_r) 
	\end{equation}

	
\end{document}
