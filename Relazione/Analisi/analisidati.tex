\documentclass[a4paper,11pt]{article}

\usepackage[italian]{babel}

\usepackage[table,xcdraw]{xcolor}

\usepackage[latin1]{inputenc}

\usepackage[T1]{fontenc}

\usepackage{graphicx}

\usepackage{indentfirst}

\usepackage{amsmath,amssymb}

\usepackage{enumitem} 

\newcommand{\virgolette}[1]{``#1''}

\usepackage[margin=1in]{geometry} %Smaller margins

\usepackage{lmodern} %Vector PDF

\usepackage{siunitx}

\usepackage{xcolor}

\usepackage{colortbl}

\usepackage{multirow}

\usepackage{rotating}

\usepackage{booktabs}

\usepackage{longtable}

\usepackage{graphicx}

\usepackage{wrapfig}


\begin{document}
\section{Raccolta e Analisi Dati}

La procedura di raccolta e di analisi dei dati per il calcolo del rapporto $e/m$ consiste principalmente nella registrazione del valore del raggio della circonferenza del fascio di elettroni \virgolette{sparato} per diversi valori di corrente e differenti differenza di potenziale. In questo modo per variazioni del campo magnetico prodotto dalle bobine di Helmoltz e del potenziale applicato al catodo che emette gli elettroni � possbile verificare la relazione lineare:
\begin{equation}\label{regressione}
(B_z R)^2 = 2\Delta V \, m/e 
\end{equation}
dove $\Delta V$ � la differenza di poetnziale applicata, $B_z$ � il capo magnetico prodotto dalle bobine di Helmoltz in funzione della corrente. Invertendo questa relazione � possibile individuare il valore di $e/m$. Questo calcolo viene fatto in due condizioni sperimentali differenti: in condizione di perpendicolarit� tra campo magnetico terrestre e campo prodotto dalle bobine, e in condizione di anti-parallelismo.

A queste procedure sperimentali � stata affiancata una seconda fase sperimentale in cui sono state effettuate diverse misure dell'intensit� del campo magnetico terrestre $B_t$, ricavato dalla relazione:
\begin{equation}\label{B_terra}
B_t= \frac{I}{I_0} (B_z \cot \theta + B_r) 
\end{equation}
dove $B_z$ e $B_r$ sono le componenti del campo magnetico generato dalle bobine dell'apparato di misura, e $\theta$ � l'angolo misurato sul goniometro
Questi dati sono necessari per una corretta valutazione dell'intensit� del campo generato dalle bobine in situazione di antiparallelismo. La presenza del campo magnetico terrestre ha infatti una influenza apprezzabile sul campo generato dalle bobine, e non pu� essere trascurato. Terremo la separazione tra misura del rapporto $e/m$ e misura del campo magnetico terrestre.

\subsection{Raccolta Dati - Misura del rapporto $e/m$}

La raccolta di dati consiste come descritto precedentemente nella regisrazione del raggio della circonferenza descritta dal fascio di elettroni deviati dal campo magnetico. In considerazione del fatto che risulta piuttoso difficile riuscire a far collimare i fermi con la posizione del raggio, anche servendosi dello specchio, si � deciso di prendere come errore per la misura del raggio $\sigma_R = \SI{5}{\milli\meter}$, valore che corrisponde a un errore percentuale dell'oridine del $4-5\%$ a seconda del raggio misurato. Di seguito le tabelle \ref{dati_perpendicolare} e \ref{dati_parallelo} riportano i valori ottenuti dei raggi, delle differenze di tensione, delle correnti applicate, e del campo magnetico\footnote{Ai valori del campo magnetico � gi� stato applicato il fattore di correzione opportuno}, con gli errori annessi su campioni di 11-13 misurazioni\footnote{Per quanto riguarda la tabella \ref{dati_parallelo}, i valori di raggio da una certa misurazione sono uguali. Questo perch�, senza perdita di precisione nella misura, si � scelto di lasciare fermi i perni dell'asta attaccata all'ampolla, e di modificare i valori di corrente e tensione fino a centrare la posizione dei fermi stessi da cui poi � stata efettuata la lettura. In questo modo si sono sveltite le procedure di misurazione senza perdita di generalit� della procedura, come mostrato dai dati raccolti.}.


% Please add the following required packages to your document preamble:
% \usepackage[table,xcdraw]{xcolor}
% If you use beamer only pass "xcolor=table" option, i.e. \documentclass[xcolor=table]{beamer}
\begin{table}[]
\centering
\caption{Dati misura $e/m$ - perpendicolare}
\label{dati_perpendicolare}
\begin{tabular}{lrrrrrrrr}
\rowcolor[HTML]{BBDAFF} 
\multicolumn{1}{c}{\cellcolor[HTML]{BBDAFF}}    & \multicolumn{1}{l}{\cellcolor[HTML]{BBDAFF}$R (\SI{}{\meter})$} & $\sigma_R  $                                         & \multicolumn{1}{l}{\cellcolor[HTML]{BBDAFF}$\Delta V (\SI{}{\volt})$} & \multicolumn{1}{l}{\cellcolor[HTML]{BBDAFF}$\sigma_V$} & \multicolumn{1}{l}{\cellcolor[HTML]{BBDAFF}$I (\SI{}{\ampere}$)} & \multicolumn{1}{l}{\cellcolor[HTML]{BBDAFF}$\sigma_I$} & \multicolumn{1}{l}{\cellcolor[HTML]{BBDAFF}$B_z(R) (\SI{}{\tesla})$} & \multicolumn{1}{l}{\cellcolor[HTML]{BBDAFF}$\sigma_B$} \\ \cline{2-9} 
\rowcolor[HTML]{C0C0C0} 
\multicolumn{1}{l|}{\cellcolor[HTML]{BBDAFF}1}  & \multicolumn{1}{r|}{\cellcolor[HTML]{C0C0C0}0.05352}              & \multicolumn{1}{r|}{\cellcolor[HTML]{C0C0C0}0.0050} & \multicolumn{1}{r|}{\cellcolor[HTML]{C0C0C0}229.2}                        & \multicolumn{1}{r|}{\cellcolor[HTML]{C0C0C0}3.0}       & \multicolumn{1}{r|}{\cellcolor[HTML]{C0C0C0}1.229}               & \multicolumn{1}{r|}{\cellcolor[HTML]{C0C0C0}0.020}     & \multicolumn{1}{r|}{\cellcolor[HTML]{C0C0C0}9.0545E-04}                  & \multicolumn{1}{r|}{\cellcolor[HTML]{C0C0C0}5.55E-06}  \\ \cline{2-9} 
\rowcolor[HTML]{EFEFEF} 
\multicolumn{1}{l|}{\cellcolor[HTML]{BBDAFF}2}  & \multicolumn{1}{r|}{\cellcolor[HTML]{EFEFEF}0.05563}              & \multicolumn{1}{r|}{\cellcolor[HTML]{EFEFEF}0.0050} & \multicolumn{1}{r|}{\cellcolor[HTML]{EFEFEF}210.5}                        & \multicolumn{1}{r|}{\cellcolor[HTML]{EFEFEF}3.0}       & \multicolumn{1}{r|}{\cellcolor[HTML]{EFEFEF}1.102}               & \multicolumn{1}{r|}{\cellcolor[HTML]{EFEFEF}0.020}     & \multicolumn{1}{r|}{\cellcolor[HTML]{EFEFEF}8.1140E-04}                  & \multicolumn{1}{r|}{\cellcolor[HTML]{EFEFEF}4.97E-06}  \\ \cline{2-9} 
\rowcolor[HTML]{C0C0C0} 
\multicolumn{1}{l|}{\cellcolor[HTML]{BBDAFF}3}  & \multicolumn{1}{r|}{\cellcolor[HTML]{C0C0C0}0.05276}              & \multicolumn{1}{r|}{\cellcolor[HTML]{C0C0C0}0.0050} & \multicolumn{1}{r|}{\cellcolor[HTML]{C0C0C0}243.3}                        & \multicolumn{1}{r|}{\cellcolor[HTML]{C0C0C0}3.0}       & \multicolumn{1}{r|}{\cellcolor[HTML]{C0C0C0}1.204}               & \multicolumn{1}{r|}{\cellcolor[HTML]{C0C0C0}0.020}     & \multicolumn{1}{r|}{\cellcolor[HTML]{C0C0C0}8.8796E-04}                  & \multicolumn{1}{r|}{\cellcolor[HTML]{C0C0C0}5.44E-06}  \\ \cline{2-9} 
\rowcolor[HTML]{EFEFEF} 
\multicolumn{1}{l|}{\cellcolor[HTML]{BBDAFF}4}  & \multicolumn{1}{r|}{\cellcolor[HTML]{EFEFEF}0.05125}              & \multicolumn{1}{r|}{\cellcolor[HTML]{EFEFEF}0.0050} & \multicolumn{1}{r|}{\cellcolor[HTML]{EFEFEF}277.3}                        & \multicolumn{1}{r|}{\cellcolor[HTML]{EFEFEF}3.0}       & \multicolumn{1}{r|}{\cellcolor[HTML]{EFEFEF}1.389}               & \multicolumn{1}{r|}{\cellcolor[HTML]{EFEFEF}0.020}     & \multicolumn{1}{r|}{\cellcolor[HTML]{EFEFEF}1.0244E-03}                  & \multicolumn{1}{r|}{\cellcolor[HTML]{EFEFEF}6.28E-06}  \\ \cline{2-9} 
\rowcolor[HTML]{C0C0C0} 
\multicolumn{1}{l|}{\cellcolor[HTML]{BBDAFF}5}  & \multicolumn{1}{r|}{\cellcolor[HTML]{C0C0C0}0.05468}              & \multicolumn{1}{r|}{\cellcolor[HTML]{C0C0C0}0.0050} & \multicolumn{1}{r|}{\cellcolor[HTML]{C0C0C0}281.7}                        & \multicolumn{1}{r|}{\cellcolor[HTML]{C0C0C0}3.0}       & \multicolumn{1}{r|}{\cellcolor[HTML]{C0C0C0}1.333}               & \multicolumn{1}{r|}{\cellcolor[HTML]{C0C0C0}0.020}     & \multicolumn{1}{r|}{\cellcolor[HTML]{C0C0C0}9.8149E-04}                  & \multicolumn{1}{r|}{\cellcolor[HTML]{C0C0C0}6.01E-06}  \\ \cline{2-9} 
\rowcolor[HTML]{EFEFEF} 
\multicolumn{1}{l|}{\cellcolor[HTML]{BBDAFF}6}  & \multicolumn{1}{r|}{\cellcolor[HTML]{EFEFEF}0.05175}              & \multicolumn{1}{r|}{\cellcolor[HTML]{EFEFEF}0.0050} & \multicolumn{1}{r|}{\cellcolor[HTML]{EFEFEF}296.2}                        & \multicolumn{1}{r|}{\cellcolor[HTML]{EFEFEF}3.0}       & \multicolumn{1}{r|}{\cellcolor[HTML]{EFEFEF}1.450}               & \multicolumn{1}{r|}{\cellcolor[HTML]{EFEFEF}0.020}     & \multicolumn{1}{r|}{\cellcolor[HTML]{EFEFEF}1.0694E-03}                  & \multicolumn{1}{r|}{\cellcolor[HTML]{EFEFEF}6.55E-06}  \\ \cline{2-9} 
\rowcolor[HTML]{C0C0C0} 
\multicolumn{1}{l|}{\cellcolor[HTML]{BBDAFF}7}  & \multicolumn{1}{r|}{\cellcolor[HTML]{C0C0C0}0.05612}              & \multicolumn{1}{r|}{\cellcolor[HTML]{C0C0C0}0.0050} & \multicolumn{1}{r|}{\cellcolor[HTML]{C0C0C0}184}                          & \multicolumn{1}{r|}{\cellcolor[HTML]{C0C0C0}3.0}       & \multicolumn{1}{r|}{\cellcolor[HTML]{C0C0C0}1.000}               & \multicolumn{1}{r|}{\cellcolor[HTML]{C0C0C0}0.020}     & \multicolumn{1}{r|}{\cellcolor[HTML]{C0C0C0}7.3586E-04}                  & \multicolumn{1}{r|}{\cellcolor[HTML]{C0C0C0}4.51E-06}  \\ \cline{2-9} 
\rowcolor[HTML]{EFEFEF} 
\multicolumn{1}{l|}{\cellcolor[HTML]{BBDAFF}8}  & \multicolumn{1}{r|}{\cellcolor[HTML]{EFEFEF}0.05984}              & \multicolumn{1}{r|}{\cellcolor[HTML]{EFEFEF}0.0050} & \multicolumn{1}{r|}{\cellcolor[HTML]{EFEFEF}199.6}                        & \multicolumn{1}{r|}{\cellcolor[HTML]{EFEFEF}3.0}       & \multicolumn{1}{r|}{\cellcolor[HTML]{EFEFEF}1.019}               & \multicolumn{1}{r|}{\cellcolor[HTML]{EFEFEF}0.020}     & \multicolumn{1}{r|}{\cellcolor[HTML]{EFEFEF}7.4773E-04}                  & \multicolumn{1}{r|}{\cellcolor[HTML]{EFEFEF}4.58E-06}  \\ \cline{2-9} 
\rowcolor[HTML]{C0C0C0} 
\multicolumn{1}{l|}{\cellcolor[HTML]{BBDAFF}9}  & \multicolumn{1}{r|}{\cellcolor[HTML]{C0C0C0}0.05277}              & \multicolumn{1}{r|}{\cellcolor[HTML]{C0C0C0}0.0050} & \multicolumn{1}{r|}{\cellcolor[HTML]{C0C0C0}256}                          & \multicolumn{1}{r|}{\cellcolor[HTML]{C0C0C0}3.0}       & \multicolumn{1}{r|}{\cellcolor[HTML]{C0C0C0}1.290}               & \multicolumn{1}{r|}{\cellcolor[HTML]{C0C0C0}0.020}     & \multicolumn{1}{r|}{\cellcolor[HTML]{C0C0C0}9.5089E-04}                  & \multicolumn{1}{r|}{\cellcolor[HTML]{C0C0C0}5.82E-06}  \\ \cline{2-9} 
\multicolumn{1}{l|}{\cellcolor[HTML]{BBDAFF}10} & \multicolumn{1}{r|}{\cellcolor[HTML]{EFEFEF}0.04653}              & \multicolumn{1}{r|}{\cellcolor[HTML]{EFEFEF}0.0050} & \multicolumn{1}{r|}{\cellcolor[HTML]{EFEFEF}190.8}                        & \multicolumn{1}{r|}{\cellcolor[HTML]{EFEFEF}3.0}       & \multicolumn{1}{r|}{\cellcolor[HTML]{EFEFEF}1.290}               & \multicolumn{1}{r|}{\cellcolor[HTML]{EFEFEF}0.020}     & \multicolumn{1}{r|}{\cellcolor[HTML]{EFEFEF}9.5370E-04}                  & \multicolumn{1}{r|}{\cellcolor[HTML]{EFEFEF}5.84E-06}  \\ \cline{2-9} 
\rowcolor[HTML]{C0C0C0} 
\multicolumn{1}{l|}{\cellcolor[HTML]{BBDAFF}11} & \multicolumn{1}{r|}{\cellcolor[HTML]{C0C0C0}0.05224}              & \multicolumn{1}{r|}{\cellcolor[HTML]{C0C0C0}0.0050} & \multicolumn{1}{r|}{\cellcolor[HTML]{C0C0C0}220.4}                        & \multicolumn{1}{r|}{\cellcolor[HTML]{C0C0C0}3.0}       & \multicolumn{1}{r|}{\cellcolor[HTML]{C0C0C0}1.210}               & \multicolumn{1}{r|}{\cellcolor[HTML]{C0C0C0}0.020}     & \multicolumn{1}{r|}{\cellcolor[HTML]{C0C0C0}8.9239E-04}                  & \multicolumn{1}{r|}{\cellcolor[HTML]{C0C0C0}5.47E-06}  \\ \cline{2-9} 

\end{tabular}
\end{table}



\begin{table}[]
\centering
\caption{Dati misura $e/m$ - antiparallelo.}
\label{dati_parallelo}
\begin{tabular}{lrrrrrrrr}
\rowcolor[HTML]{BBDAFF} 
\multicolumn{1}{c}{\cellcolor[HTML]{BBDAFF}}    & \multicolumn{1}{l}{\cellcolor[HTML]{BBDAFF}$R (\SI{}{\meter})$} & $\sigma_R       $                                    & \multicolumn{1}{l}{\cellcolor[HTML]{BBDAFF}$\Delta V (\SI{}{\volt})$} & \multicolumn{1}{l}{\cellcolor[HTML]{BBDAFF}$\sigma_V$} & \multicolumn{1}{l}{\cellcolor[HTML]{BBDAFF}$I (\SI{}{\ampere})$} & \multicolumn{1}{l}{\cellcolor[HTML]{BBDAFF}$\sigma_I$} & \multicolumn{1}{l}{\cellcolor[HTML]{BBDAFF}$B_z(R) (\SI{}{\tesla})$} & \multicolumn{1}{l}{\cellcolor[HTML]{BBDAFF}$\sigma_B$} \\ \cline{2-9} 
\rowcolor[HTML]{C0C0C0} 
\multicolumn{1}{l|}{\cellcolor[HTML]{BBDAFF}1}  & \multicolumn{1}{r|}{\cellcolor[HTML]{C0C0C0}0.05023}              & \multicolumn{1}{r|}{\cellcolor[HTML]{C0C0C0}0.0050} & \multicolumn{1}{r|}{\cellcolor[HTML]{C0C0C0}179.5}                        & \multicolumn{1}{r|}{\cellcolor[HTML]{C0C0C0}3.0}       & \multicolumn{1}{r|}{\cellcolor[HTML]{C0C0C0}1.090}               & \multicolumn{1}{r|}{\cellcolor[HTML]{C0C0C0}0.020}     & \multicolumn{1}{r|}{\cellcolor[HTML]{C0C0C0}7.7522E-04}                  & \multicolumn{1}{r|}{\cellcolor[HTML]{C0C0C0}4.93E-06}  \\ \cline{2-9} 
\rowcolor[HTML]{EFEFEF} 
\multicolumn{1}{l|}{\cellcolor[HTML]{BBDAFF}2}  & \multicolumn{1}{r|}{\cellcolor[HTML]{EFEFEF}0.05362}              & \multicolumn{1}{r|}{\cellcolor[HTML]{EFEFEF}0.0050} & \multicolumn{1}{r|}{\cellcolor[HTML]{EFEFEF}190.4}                        & \multicolumn{1}{r|}{\cellcolor[HTML]{EFEFEF}3.0}       & \multicolumn{1}{r|}{\cellcolor[HTML]{EFEFEF}1.042}               & \multicolumn{1}{r|}{\cellcolor[HTML]{EFEFEF}0.020}     & \multicolumn{1}{r|}{\cellcolor[HTML]{EFEFEF}7.3828E-04}                  & \multicolumn{1}{r|}{\cellcolor[HTML]{EFEFEF}4.7E-06}   \\ \cline{2-9} 
\rowcolor[HTML]{C0C0C0} 
\multicolumn{1}{l|}{\cellcolor[HTML]{BBDAFF}3}  & \multicolumn{1}{r|}{\cellcolor[HTML]{C0C0C0}0.05110}              & \multicolumn{1}{r|}{\cellcolor[HTML]{C0C0C0}0.0050} & \multicolumn{1}{r|}{\cellcolor[HTML]{C0C0C0}200}                          & \multicolumn{1}{r|}{\cellcolor[HTML]{C0C0C0}3.0}       & \multicolumn{1}{r|}{\cellcolor[HTML]{C0C0C0}1.073}               & \multicolumn{1}{r|}{\cellcolor[HTML]{C0C0C0}0.020}     & \multicolumn{1}{r|}{\cellcolor[HTML]{C0C0C0}7.6231E-04}                  & \multicolumn{1}{r|}{\cellcolor[HTML]{C0C0C0}4.85E-06}  \\ \cline{2-9} 
\rowcolor[HTML]{EFEFEF} 
\multicolumn{1}{l|}{\cellcolor[HTML]{BBDAFF}4}  & \multicolumn{1}{r|}{\cellcolor[HTML]{EFEFEF}0.05110}              & \multicolumn{1}{r|}{\cellcolor[HTML]{EFEFEF}0.0050} & \multicolumn{1}{r|}{\cellcolor[HTML]{EFEFEF}208}                          & \multicolumn{1}{r|}{\cellcolor[HTML]{EFEFEF}3.0}       & \multicolumn{1}{r|}{\cellcolor[HTML]{EFEFEF}1.210}               & \multicolumn{1}{r|}{\cellcolor[HTML]{EFEFEF}0.020}     & \multicolumn{1}{r|}{\cellcolor[HTML]{EFEFEF}8.6339E-04}                  & \multicolumn{1}{r|}{\cellcolor[HTML]{EFEFEF}5.47E-06}  \\ \cline{2-9} 
\rowcolor[HTML]{C0C0C0} 
\multicolumn{1}{l|}{\cellcolor[HTML]{BBDAFF}5}  & \multicolumn{1}{r|}{\cellcolor[HTML]{C0C0C0}0.05110}              & \multicolumn{1}{r|}{\cellcolor[HTML]{C0C0C0}0.0050} & \multicolumn{1}{r|}{\cellcolor[HTML]{C0C0C0}222.5}                        & \multicolumn{1}{r|}{\cellcolor[HTML]{C0C0C0}3.0}       & \multicolumn{1}{r|}{\cellcolor[HTML]{C0C0C0}1.303}               & \multicolumn{1}{r|}{\cellcolor[HTML]{C0C0C0}0.020}     & \multicolumn{1}{r|}{\cellcolor[HTML]{C0C0C0}9.3201E-04}                  & \multicolumn{1}{r|}{\cellcolor[HTML]{C0C0C0}5.89E-06}  \\ \cline{2-9} 
\rowcolor[HTML]{EFEFEF} 
\multicolumn{1}{l|}{\cellcolor[HTML]{BBDAFF}6}  & \multicolumn{1}{r|}{\cellcolor[HTML]{EFEFEF}0.05110}              & \multicolumn{1}{r|}{\cellcolor[HTML]{EFEFEF}0.0050} & \multicolumn{1}{r|}{\cellcolor[HTML]{EFEFEF}230}                          & \multicolumn{1}{r|}{\cellcolor[HTML]{EFEFEF}3.0}       & \multicolumn{1}{r|}{\cellcolor[HTML]{EFEFEF}1.301}               & \multicolumn{1}{r|}{\cellcolor[HTML]{EFEFEF}0.020}     & \multicolumn{1}{r|}{\cellcolor[HTML]{EFEFEF}9.3054E-04}                  & \multicolumn{1}{r|}{\cellcolor[HTML]{EFEFEF}5.88E-06}  \\ \cline{2-9} 
\rowcolor[HTML]{C0C0C0} 
\multicolumn{1}{l|}{\cellcolor[HTML]{BBDAFF}7}  & \multicolumn{1}{r|}{\cellcolor[HTML]{C0C0C0}0.05110}              & \multicolumn{1}{r|}{\cellcolor[HTML]{C0C0C0}0.0050} & \multicolumn{1}{r|}{\cellcolor[HTML]{C0C0C0}241.6}                        & \multicolumn{1}{r|}{\cellcolor[HTML]{C0C0C0}3.0}       & \multicolumn{1}{r|}{\cellcolor[HTML]{C0C0C0}1.302}               & \multicolumn{1}{r|}{\cellcolor[HTML]{C0C0C0}0.020}     & \multicolumn{1}{r|}{\cellcolor[HTML]{C0C0C0}9.3128E-04}                  & \multicolumn{1}{r|}{\cellcolor[HTML]{C0C0C0}5.88E-06}  \\ \cline{2-9} 
\rowcolor[HTML]{EFEFEF} 
\multicolumn{1}{l|}{\cellcolor[HTML]{BBDAFF}8}  & \multicolumn{1}{r|}{\cellcolor[HTML]{EFEFEF}0.05110}              & \multicolumn{1}{r|}{\cellcolor[HTML]{EFEFEF}0.0050} & \multicolumn{1}{r|}{\cellcolor[HTML]{EFEFEF}249.9}                        & \multicolumn{1}{r|}{\cellcolor[HTML]{EFEFEF}3.0}       & \multicolumn{1}{r|}{\cellcolor[HTML]{EFEFEF}1.354}               & \multicolumn{1}{r|}{\cellcolor[HTML]{EFEFEF}0.020}     & \multicolumn{1}{r|}{\cellcolor[HTML]{EFEFEF}9.6964E-04}                  & \multicolumn{1}{r|}{\cellcolor[HTML]{EFEFEF}6.12E-06}  \\ \cline{2-9} 
\rowcolor[HTML]{C0C0C0} 
\multicolumn{1}{l|}{\cellcolor[HTML]{BBDAFF}9}  & \multicolumn{1}{r|}{\cellcolor[HTML]{C0C0C0}0.05110}              & \multicolumn{1}{r|}{\cellcolor[HTML]{C0C0C0}0.0050} & \multicolumn{1}{r|}{\cellcolor[HTML]{C0C0C0}260.9}                        & \multicolumn{1}{r|}{\cellcolor[HTML]{C0C0C0}3.0}       & \multicolumn{1}{r|}{\cellcolor[HTML]{C0C0C0}1.315}               & \multicolumn{1}{r|}{\cellcolor[HTML]{C0C0C0}0.020}     & \multicolumn{1}{r|}{\cellcolor[HTML]{C0C0C0}9.4087E-04}                  & \multicolumn{1}{r|}{\cellcolor[HTML]{C0C0C0}5.94E-06}  \\ \cline{2-9} 
\rowcolor[HTML]{EFEFEF} 
\multicolumn{1}{l|}{\cellcolor[HTML]{BBDAFF}10} & \multicolumn{1}{r|}{\cellcolor[HTML]{EFEFEF}0.05110}              & \multicolumn{1}{r|}{\cellcolor[HTML]{EFEFEF}0.0050} & \multicolumn{1}{r|}{\cellcolor[HTML]{EFEFEF}279.9}                        & \multicolumn{1}{r|}{\cellcolor[HTML]{EFEFEF}3.0}       & \multicolumn{1}{r|}{\cellcolor[HTML]{EFEFEF}1.407}               & \multicolumn{1}{r|}{\cellcolor[HTML]{EFEFEF}0.020}     & \multicolumn{1}{r|}{\cellcolor[HTML]{EFEFEF}1.0088E-03}                  & \multicolumn{1}{r|}{\cellcolor[HTML]{EFEFEF}6.36E-06}  \\ \cline{2-9} 
\rowcolor[HTML]{C0C0C0} 
\multicolumn{1}{l|}{\cellcolor[HTML]{BBDAFF}11} & \multicolumn{1}{r|}{\cellcolor[HTML]{C0C0C0}0.05110}              & \multicolumn{1}{r|}{\cellcolor[HTML]{C0C0C0}0.0050} & \multicolumn{1}{r|}{\cellcolor[HTML]{C0C0C0}292.2}                        & \multicolumn{1}{r|}{\cellcolor[HTML]{C0C0C0}3.0}       & \multicolumn{1}{r|}{\cellcolor[HTML]{C0C0C0}1.465}               & \multicolumn{1}{r|}{\cellcolor[HTML]{C0C0C0}0.020}     & \multicolumn{1}{r|}{\cellcolor[HTML]{C0C0C0}1.0515E-03}                  & \multicolumn{1}{r|}{\cellcolor[HTML]{C0C0C0}6.62E-06}  \\ \cline{2-9} 
\rowcolor[HTML]{EFEFEF} 
\multicolumn{1}{l|}{\cellcolor[HTML]{BBDAFF}12} & \multicolumn{1}{r|}{\cellcolor[HTML]{EFEFEF}0.05110}              & \multicolumn{1}{r|}{\cellcolor[HTML]{EFEFEF}0.0050} & \multicolumn{1}{r|}{\cellcolor[HTML]{EFEFEF}271}                          & \multicolumn{1}{r|}{\cellcolor[HTML]{EFEFEF}3.0}       & \multicolumn{1}{r|}{\cellcolor[HTML]{EFEFEF}1.382}               & \multicolumn{1}{r|}{\cellcolor[HTML]{EFEFEF}0.020}     & \multicolumn{1}{r|}{\cellcolor[HTML]{EFEFEF}9.9030E-04}                  & \multicolumn{1}{r|}{\cellcolor[HTML]{EFEFEF}6.25E-06}  \\ \cline{2-9} 
\rowcolor[HTML]{C0C0C0} 
\multicolumn{1}{l|}{\cellcolor[HTML]{BBDAFF}13} & \multicolumn{1}{r|}{\cellcolor[HTML]{C0C0C0}0.05110}              & \multicolumn{1}{r|}{\cellcolor[HTML]{C0C0C0}0.0050} & \multicolumn{1}{r|}{\cellcolor[HTML]{C0C0C0}299.8}                        & \multicolumn{1}{r|}{\cellcolor[HTML]{C0C0C0}3.0}       & \multicolumn{1}{r|}{\cellcolor[HTML]{C0C0C0}1.452}               & \multicolumn{1}{r|}{\cellcolor[HTML]{C0C0C0}0.020}     & \multicolumn{1}{r|}{\cellcolor[HTML]{C0C0C0}1.0420E-03}                  & \multicolumn{1}{r|}{\cellcolor[HTML]{C0C0C0}6.56E-06}  \\ \cline{2-9} 
\end{tabular}
\end{table}
Se per gli errori relativi a $\Delta V, I$ e $R$ si sono tenuti errori costanti il calcolo di $B_z$:
\[
B_z = \mu_0 \frac{8}{5\sqrt{5}} \frac{NI}{R_b}
\] 
mostra dipendenza sia dal valore della corrente che da quello del raggio delle bobine, la cui misura � stata di $\SI{0.1575}{\meter}$ con un errore di $\SI{0.0009}{\meter}$ -- questi dati sono il risultato di analisi statistica su pi� misure del raggio delle bobine, che � identico alla distanza tra le stesse. In conseguenza della doppia dipendenza di $B_z$ si � applicata la formula di propagazione degli errori:
\[
\sigma_B = \sqrt{\left(\frac{\partial B}{\partial R_b} \sigma_R \right)^2 + \left(\frac{\partial B}{\partial I} \sigma_I \right)^2}
\] 
che applicata caso per caso d� i risultati riportati in tabella.

\'E bene far notare da subito un fatto relativo alla tabella \ref{dati_parallelo}: nella colonna reltiva ai valori del campo magnetico si � tenuto conto non solo dei fattori di correzione opportuni (indicati nell tabella delle dispense), ma al valore ottenuto per $B_z$ � stato sottratto il campo magnetico terrestre, il cui valore e giustificazione della misura � data nella sezione che segue\footnote{Questa scelta dipende dal fatto di non voler appesantire eccessivamente le tabelle che vengono riportate, facilitando per quanto possibile la lettura dei dati. Inoltre in questo modo � possibile avere continuit� tra questi dat e quelli che vengono riportati nell'esecuzione delle regressioni lineari.}.

\subsection{Raccolta e Analisi Dati- Misura del Campo Magnetico $B_t$}

Come detto questa procedura � necessaria per la correzione dei valori di $B_z$ coerentemente con quanto fatto nella tabella \ref{dati_parallelo}. La procedura di analisi per i dati raccolti -- che sono riportati nella tabella \ref{dati_cmt}, consiste nella verifica della consistenza dei valori del campo magnetico terrestre ottenuti per differenti angoli, tramite la formula \ref{B_terra} con i relativi errori. Questa procedura � stata ottenuta tramite una media pesata dei termini considerati.

\begin{table}[]
\centering
\caption{Dati raccolti per il calcolo del Campo Magnetico Terrestre}
\label{dati_cmt}
\begin{tabular}{rrrrrrr}
\rowcolor[HTML]{BBDAFF} 
\multicolumn{1}{l}{\cellcolor[HTML]{BBDAFF}$I (\SI{}{\ampere})$} & $B_z (\SI{}{\tesla})$                                & \multicolumn{1}{l}{\cellcolor[HTML]{BBDAFF}$B_r(\SI{}{\tesla})$} & \multicolumn{1}{l}{\cellcolor[HTML]{BBDAFF}$\theta$(rad)} & \multicolumn{1}{l}{\cellcolor[HTML]{BBDAFF}$\sigma_{\theta}$} & \multicolumn{1}{l}{\cellcolor[HTML]{BBDAFF}$B_t (\SI{}{\tesla})$} & \multicolumn{1}{l}{\cellcolor[HTML]{BBDAFF}$\sigma_B$}   \\ \hline
\rowcolor[HTML]{C0C0C0} 
\multicolumn{1}{|r|}{\cellcolor[HTML]{C0C0C0}0.016}            & \multicolumn{1}{r|}{\cellcolor[HTML]{C0C0C0}15.379} & \multicolumn{1}{r|}{\cellcolor[HTML]{C0C0C0}2.86E-02}            & \multicolumn{1}{r|}{\cellcolor[HTML]{C0C0C0}0.6981}      & \multicolumn{1}{r|}{\cellcolor[HTML]{C0C0C0}0.0175}         & \multicolumn{1}{r|}{\cellcolor[HTML]{C0C0C0}2.978E-01}            & \multicolumn{1}{r|}{\cellcolor[HTML]{C0C0C0}2.133E-02} \\ \hline
\rowcolor[HTML]{EFEFEF} 
\multicolumn{1}{|r|}{\cellcolor[HTML]{EFEFEF}0.013}            & \multicolumn{1}{r|}{\cellcolor[HTML]{EFEFEF}1.5295} & \multicolumn{1}{r|}{\cellcolor[HTML]{EFEFEF}3.33E-02}            & \multicolumn{1}{r|}{\cellcolor[HTML]{EFEFEF}0.6109}      & \multicolumn{1}{r|}{\cellcolor[HTML]{EFEFEF}0.0175}         & \multicolumn{1}{r|}{\cellcolor[HTML]{EFEFEF}2.883E-01}            & \multicolumn{1}{r|}{\cellcolor[HTML]{EFEFEF}2.457E-02} \\ \hline
\rowcolor[HTML]{C0C0C0} 
\multicolumn{1}{|r|}{\cellcolor[HTML]{C0C0C0}0.011}            & \multicolumn{1}{r|}{\cellcolor[HTML]{C0C0C0}1.519}  & \multicolumn{1}{r|}{\cellcolor[HTML]{C0C0C0}3.62E-02}            & \multicolumn{1}{r|}{\cellcolor[HTML]{C0C0C0}0.5236}      & \multicolumn{1}{r|}{\cellcolor[HTML]{C0C0C0}0.0175}         & \multicolumn{1}{r|}{\cellcolor[HTML]{C0C0C0}2.934E-01}            & \multicolumn{1}{r|}{\cellcolor[HTML]{C0C0C0}2.912E-02} \\ \hline
\rowcolor[HTML]{EFEFEF} 
\multicolumn{1}{|r|}{\cellcolor[HTML]{EFEFEF}0.023}            & \multicolumn{1}{r|}{\cellcolor[HTML]{EFEFEF}1.5471} & \multicolumn{1}{r|}{\cellcolor[HTML]{EFEFEF}1.67E-02}            & \multicolumn{1}{r|}{\cellcolor[HTML]{EFEFEF}0.8727}      & \multicolumn{1}{r|}{\cellcolor[HTML]{EFEFEF}0.0175}         & \multicolumn{1}{r|}{\cellcolor[HTML]{EFEFEF}3.024E-01}            & \multicolumn{1}{r|}{\cellcolor[HTML]{EFEFEF}1.690E-02} \\ \hline
\rowcolor[HTML]{C0C0C0} 
\multicolumn{1}{|r|}{\cellcolor[HTML]{C0C0C0}0.032}            & \multicolumn{1}{r|}{\cellcolor[HTML]{C0C0C0}1.5472} & \multicolumn{1}{r|}{\cellcolor[HTML]{C0C0C0}6.20E-03}            & \multicolumn{1}{r|}{\cellcolor[HTML]{C0C0C0}1.0472}      & \multicolumn{1}{r|}{\cellcolor[HTML]{C0C0C0}0.0175}         & \multicolumn{1}{r|}{\cellcolor[HTML]{C0C0C0}2.878E-01}            & \multicolumn{1}{r|}{\cellcolor[HTML]{C0C0C0}1.464E-02} \\ \hline
\rowcolor[HTML]{EFEFEF} 
\multicolumn{1}{|r|}{\cellcolor[HTML]{EFEFEF}0.053}            & \multicolumn{1}{r|}{\cellcolor[HTML]{EFEFEF}1.5423} & \multicolumn{1}{r|}{\cellcolor[HTML]{EFEFEF}1.00E-04}            & \multicolumn{1}{r|}{\cellcolor[HTML]{EFEFEF}1.2217}      & \multicolumn{1}{r|}{\cellcolor[HTML]{EFEFEF}0.0175}         & \multicolumn{1}{r|}{\cellcolor[HTML]{EFEFEF}2.976E-01}            & \multicolumn{1}{r|}{\cellcolor[HTML]{EFEFEF}1.715E-02} \\ \hline
\rowcolor[HTML]{C0C0C0} 
\multicolumn{1}{|r|}{\cellcolor[HTML]{C0C0C0}0.007}            & \multicolumn{1}{r|}{\cellcolor[HTML]{C0C0C0}1.4951} & \multicolumn{1}{r|}{\cellcolor[HTML]{C0C0C0}3.42E-02}            & \multicolumn{1}{r|}{\cellcolor[HTML]{C0C0C0}0.3491}      & \multicolumn{1}{r|}{\cellcolor[HTML]{C0C0C0}0.0175}         & \multicolumn{1}{r|}{\cellcolor[HTML]{C0C0C0}2.899E-01}            & \multicolumn{1}{r|}{\cellcolor[HTML]{C0C0C0}4.428E-02} \\ \hline
\end{tabular}
\end{table}

Da questi dati si pu� osservare che la compatibilit� tra i valori ottenuti -- supponendo un errore costante sul valore della corrente di $\SI{0.001}{\ampere}$ � verificato e dalla procedura di media pesata si ottiene dunque $B_t = \SI{2.945e-1}{\tesla}$ con un errore di $\SI{8e-4}{\tesla}$. I valori parziali di compatiblit� tra le misure effettuate sono riportati nella tabella \ref{comp_cmt}.


\begin{table}[]
\centering
\caption{Valori di compatibilit� per le misure del campo magnetico terrestre.}
\label{comp_cmt}
\begin{tabular}{rrrrrrrlllll}
\rowcolor[HTML]{BBDAFF} 
\multicolumn{1}{l}{\cellcolor[HTML]{BBDAFF}comp.}     & mis.                                             & \multicolumn{1}{l}{\cellcolor[HTML]{BBDAFF}comp.}    & \multicolumn{1}{l}{\cellcolor[HTML]{BBDAFF}mis.} & \multicolumn{1}{l}{\cellcolor[HTML]{BBDAFF}comp.}    & \multicolumn{1}{l}{\cellcolor[HTML]{BBDAFF}mis.} & \multicolumn{1}{l}{\cellcolor[HTML]{BBDAFF}comp.}    & mis.                                             & comp.                                                & mis.                                             & comp.                                                & mis.                                             \\ \hline
\rowcolor[HTML]{C0C0C0} 
\multicolumn{1}{|r|}{\cellcolor[HTML]{C0C0C0}2.93E-1} & \multicolumn{1}{r|}{\cellcolor[HTML]{C0C0C0}1-2} & \multicolumn{1}{r|}{\cellcolor[HTML]{C0C0C0}1.33E-1} & \multicolumn{1}{r|}{\cellcolor[HTML]{C0C0C0}2-3} & \multicolumn{1}{r|}{\cellcolor[HTML]{C0C0C0}2.68E-1} & \multicolumn{1}{r|}{\cellcolor[HTML]{C0C0C0}3-4} & \multicolumn{1}{r|}{\cellcolor[HTML]{C0C0C0}6.52E-1} & \multicolumn{1}{l|}{\cellcolor[HTML]{C0C0C0}4-5} & \multicolumn{1}{l|}{\cellcolor[HTML]{C0C0C0}4.31E-1} & \multicolumn{1}{l|}{\cellcolor[HTML]{C0C0C0}5-6} & \multicolumn{1}{l|}{\cellcolor[HTML]{C0C0C0}1.61E-1} & \multicolumn{1}{l|}{\cellcolor[HTML]{C0C0C0}6-7} \\ \hline
\rowcolor[HTML]{EFEFEF} 
\multicolumn{1}{|r|}{\cellcolor[HTML]{EFEFEF}1.23E-1} & \multicolumn{1}{r|}{\cellcolor[HTML]{EFEFEF}1-3} & \multicolumn{1}{r|}{\cellcolor[HTML]{EFEFEF}4.73E-1} & \multicolumn{1}{r|}{\cellcolor[HTML]{EFEFEF}2-4} & \multicolumn{1}{r|}{\cellcolor[HTML]{EFEFEF}1.70E-1} & \multicolumn{1}{r|}{\cellcolor[HTML]{EFEFEF}3-5} & \multicolumn{1}{r|}{\cellcolor[HTML]{EFEFEF}2.01E-1} & \multicolumn{1}{l|}{\cellcolor[HTML]{EFEFEF}4-6} & \multicolumn{1}{l|}{\cellcolor[HTML]{EFEFEF}4.51E-2} & \multicolumn{1}{l|}{\cellcolor[HTML]{EFEFEF}5-7} & \multicolumn{1}{l|}{\cellcolor[HTML]{EFEFEF}}        & \multicolumn{1}{l|}{\cellcolor[HTML]{EFEFEF}}    \\ \hline
\rowcolor[HTML]{C0C0C0} 
\multicolumn{1}{|r|}{\cellcolor[HTML]{C0C0C0}1.69E-1} & \multicolumn{1}{r|}{\cellcolor[HTML]{C0C0C0}1-4} & \multicolumn{1}{r|}{\cellcolor[HTML]{C0C0C0}1.61E-2} & \multicolumn{1}{r|}{\cellcolor[HTML]{C0C0C0}2-5} & \multicolumn{1}{r|}{\cellcolor[HTML]{C0C0C0}1.23E-1} & \multicolumn{1}{r|}{\cellcolor[HTML]{C0C0C0}3-6} & \multicolumn{1}{r|}{\cellcolor[HTML]{C0C0C0}2.63E-1} & \multicolumn{1}{l|}{\cellcolor[HTML]{C0C0C0}4-7} & \multicolumn{1}{l|}{\cellcolor[HTML]{C0C0C0}}        & \multicolumn{1}{l|}{\cellcolor[HTML]{C0C0C0}}    & \multicolumn{1}{l|}{\cellcolor[HTML]{C0C0C0}}        & \multicolumn{1}{l|}{\cellcolor[HTML]{C0C0C0}}    \\ \hline
\rowcolor[HTML]{EFEFEF} 
\multicolumn{1}{|r|}{\cellcolor[HTML]{EFEFEF}3.86E-1} & \multicolumn{1}{r|}{\cellcolor[HTML]{EFEFEF}1-5} & \multicolumn{1}{r|}{\cellcolor[HTML]{EFEFEF}3.09E-1} & \multicolumn{1}{r|}{\cellcolor[HTML]{EFEFEF}2-6} & \multicolumn{1}{r|}{\cellcolor[HTML]{EFEFEF}6.51E-2} & \multicolumn{1}{r|}{\cellcolor[HTML]{EFEFEF}3-7} & \multicolumn{1}{r|}{\cellcolor[HTML]{EFEFEF}}        & \multicolumn{1}{l|}{\cellcolor[HTML]{EFEFEF}}    & \multicolumn{1}{l|}{\cellcolor[HTML]{EFEFEF}}        & \multicolumn{1}{l|}{\cellcolor[HTML]{EFEFEF}}    & \multicolumn{1}{l|}{\cellcolor[HTML]{EFEFEF}}        & \multicolumn{1}{l|}{\cellcolor[HTML]{EFEFEF}}    \\ \hline
\rowcolor[HTML]{C0C0C0} 
\multicolumn{1}{|r|}{\cellcolor[HTML]{C0C0C0}9.30E-3} & \multicolumn{1}{r|}{\cellcolor[HTML]{C0C0C0}1-6} & \multicolumn{1}{r|}{\cellcolor[HTML]{C0C0C0}3.24E-2} & \multicolumn{1}{r|}{\cellcolor[HTML]{C0C0C0}2-7} & \multicolumn{1}{r|}{\cellcolor[HTML]{C0C0C0}}        & \multicolumn{1}{r|}{\cellcolor[HTML]{C0C0C0}}    & \multicolumn{1}{r|}{\cellcolor[HTML]{C0C0C0}}        & \multicolumn{1}{l|}{\cellcolor[HTML]{C0C0C0}}    & \multicolumn{1}{l|}{\cellcolor[HTML]{C0C0C0}}        & \multicolumn{1}{l|}{\cellcolor[HTML]{C0C0C0}}    & \multicolumn{1}{l|}{\cellcolor[HTML]{C0C0C0}}        & \multicolumn{1}{l|}{\cellcolor[HTML]{C0C0C0}}    \\ \hline
\rowcolor[HTML]{EFEFEF} 
\multicolumn{1}{|r|}{\cellcolor[HTML]{EFEFEF}1.60E-1} & \multicolumn{1}{r|}{\cellcolor[HTML]{EFEFEF}1-7} & \multicolumn{1}{r|}{\cellcolor[HTML]{EFEFEF}}        & \multicolumn{1}{r|}{\cellcolor[HTML]{EFEFEF}}    & \multicolumn{1}{r|}{\cellcolor[HTML]{EFEFEF}}        & \multicolumn{1}{r|}{\cellcolor[HTML]{EFEFEF}}    & \multicolumn{1}{r|}{\cellcolor[HTML]{EFEFEF}}        & \multicolumn{1}{l|}{\cellcolor[HTML]{EFEFEF}}    & \multicolumn{1}{l|}{\cellcolor[HTML]{EFEFEF}}        & \multicolumn{1}{l|}{\cellcolor[HTML]{EFEFEF}}    & \multicolumn{1}{l|}{\cellcolor[HTML]{EFEFEF}}        & \multicolumn{1}{l|}{\cellcolor[HTML]{EFEFEF}}    \\ \hline
\end{tabular}
\end{table}


\section{Analisi dei Dati per la determinazione del rapporto $e/m$}

Una volta determinato il valore del campo magnetico terrestre con il suo errore, � possibile ricavare i dati per la verifica della relazione lineare \ref{regressione}. In particolare il valore di $B_t$ influenza le misure soprattutto nel caso di disposizione antiparallela della strumentazione rispetto al campo magnetico, come detto in precedenza.

Nelle tabelle \ref{retta_perp} e \ref{retta_parall} sono riportati i valori su cui applicare l'algoritmo di regressione lineare. Si � verificato in particolare che l'errore sulle ascisse -- nel caso particolare $2\Delta V$, fosse di ordine inferiore rispetto a quello sulle ordinate e potesse essere trascurato\footnote{Si � osservato che la somma in quadratura degli errori, moltiplicando l'errore sulle ascisse per il coefficiente angolare della retta individuata aveva un inferiore all'$1\%$ dell'errore totale, cio� di circa lo $0.1\%$ del valore dell'ordinata.}. 


\begin{table}[]
\centering
\caption{Dati per la regressione - perpendicolare.}
\label{retta_perp}
\begin{tabular}{lllll}
\rowcolor[HTML]{BBDAFF} 
                           & $2\Delta V (\SI{}{\volt})$ & $\sigma_V$ & $(B R)^2 (\SI{}{\tesla}^2\SI{}{\meter}^2)$ & $\sigma_{BR}$  \\
\rowcolor[HTML]{C0C0C0} 
\cellcolor[HTML]{BBDAFF}1  & 458.4          & 6          & 2.348E-09                                                & 4.40E-10  \\
\rowcolor[HTML]{EFEFEF} 
\cellcolor[HTML]{BBDAFF}2  & 421            & 6          & 2.038E-09                                                & 3.67E-10  \\
\rowcolor[HTML]{C0C0C0} 
\cellcolor[HTML]{BBDAFF}3  & 486.6          & 6          & 2.195E-09                                                & 4.17E-10  \\
\rowcolor[HTML]{EFEFEF} 
\cellcolor[HTML]{BBDAFF}4  & 554.6          & 6          & 2.756E-09                                                & 5.39E-10  \\
\rowcolor[HTML]{C0C0C0} 
\cellcolor[HTML]{BBDAFF}5  & 563.4          & 6          & 2.880E-09                                                & 5.28E-10  \\
\rowcolor[HTML]{EFEFEF} 
\cellcolor[HTML]{BBDAFF}6  & 592.4          & 6          & 3.062E-09                                                & 5.93E-10  \\
\rowcolor[HTML]{C0C0C0} 
\cellcolor[HTML]{BBDAFF}7  & 368            & 6          & 1.706E-09                                                & 3.05E-10  \\
\rowcolor[HTML]{EFEFEF} 
\cellcolor[HTML]{BBDAFF}8  & 399.2          & 6          & 2.002E-09                                                & 3.35E-10  \\
\rowcolor[HTML]{C0C0C0} 
\cellcolor[HTML]{BBDAFF}9  & 512            & 6          & 2.518E-09                                                & 4.78E-10  \\
\rowcolor[HTML]{EFEFEF} 
\cellcolor[HTML]{BBDAFF}10 & 381.6          & 6          & 1.969E-09                                                & 4.24E-10  \\
\rowcolor[HTML]{C0C0C0} 
\cellcolor[HTML]{BBDAFF}11 & 440.8          & 6          & 2.173E-09                                                & 4.17E-10  
\end{tabular}
\end{table}

\begin{table}[]
\centering
\caption{Dati per la regressione - antiparallelo.}
\label{retta_parall}
\begin{tabular}{rrrrr}
\rowcolor[HTML]{BBDAFF} 
\multicolumn{1}{l}{\cellcolor[HTML]{BBDAFF}} & $2\Delta V (\SI{}{\volt})$ & \multicolumn{1}{l}{\cellcolor[HTML]{BBDAFF}$\sigma_V$} & \multicolumn{1}{l}{\cellcolor[HTML]{BBDAFF}$(B R)^2 (\SI{}{\tesla}^2\SI{}{\meter}^2)$} & \multicolumn{1}{l}{\cellcolor[HTML]{BBDAFF}$\sigma_{BR}$} \\
\rowcolor[HTML]{C0C0C0} 
\cellcolor[HTML]{BBDAFF}1                    & 359                        & 6                                                      & 1.516E-09                                                                              & 3.02E-10                                                  \\
\rowcolor[HTML]{EFEFEF} 
\cellcolor[HTML]{BBDAFF}2                    & 380.8                      & 6                                                      & 1.567E-09                                                                              & 2.93E-10                                                  \\
\rowcolor[HTML]{C0C0C0} 
\cellcolor[HTML]{BBDAFF}3                    & 400                        & 6                                                      & 1.518E-09                                                                              & 2.98E-10                                                  \\
\rowcolor[HTML]{EFEFEF} 
\cellcolor[HTML]{BBDAFF}4                    & 416                        & 6                                                      & 1.947E-09                                                                              & 3.82E-10                                                  \\
\rowcolor[HTML]{C0C0C0} 
\cellcolor[HTML]{BBDAFF}5                    & 445                        & 6                                                      & 2.269E-09                                                                              & 4.45E-10                                                  \\
\rowcolor[HTML]{EFEFEF} 
\cellcolor[HTML]{BBDAFF}6                    & 460                        & 6                                                      & 2.261E-09                                                                              & 4.43E-10                                                  \\
\rowcolor[HTML]{C0C0C0} 
\cellcolor[HTML]{BBDAFF}7                    & 483.2                      & 6                                                      & 2.265E-09                                                                              & 4.44E-10                                                  \\
\rowcolor[HTML]{EFEFEF} 
\cellcolor[HTML]{BBDAFF}8                    & 499.8                      & 6                                                      & 2.455E-09                                                                              & 4.81E-10                                                  \\
\rowcolor[HTML]{C0C0C0} 
\cellcolor[HTML]{BBDAFF}9                    & 521.8                      & 6                                                      & 2.312E-09                                                                              & 4.53E-10                                                  \\
\rowcolor[HTML]{EFEFEF} 
\cellcolor[HTML]{BBDAFF}10                   & 559.8                      & 6                                                      & 2.657E-09                                                                              & 5.21E-10                                                  \\
\rowcolor[HTML]{C0C0C0} 
\cellcolor[HTML]{BBDAFF}11                   & 584.4                      & 6                                                      & 2.888E-09                                                                              & 5.66E-10                                                  \\
\rowcolor[HTML]{EFEFEF} 
\cellcolor[HTML]{BBDAFF}12                   & 542                        & 6                                                      & 2.561E-09                                                                              & 5.02E-10                                                  \\
\rowcolor[HTML]{C0C0C0} 
\cellcolor[HTML]{BBDAFF}13                   & 599.6                      & 6                                                      & 2.835E-09                                                                              & 5.56E-10                                                 
\end{tabular}
\end{table}

Alla luce del tipo di dati ricavato, con errore costante e trascurabile sulle ascisse e errore variabile sulle ordinate, si � scelto di applicare l'algoritmo di regressione lineare pesata. Nel caso particolare i pesi sono ricavati dall'inverso dei quadrati dei valori di $\sigma_{BR}$. I valori per coefficiente angolare e intercetta ottenuti nel caso \virgolette{perpendicolare} sono:
\begin{equation}\label{mq_perp}
 m/e = 5.337E-12 \pm 4.42E-13 \; \; q= -1.85E-10 \pm 1.96E-10
\end{equation}
da cui � possibile immediatamente osservare la compatibilit� con l'origine dell'intercetta.

Per quanto riguarda invece il caso \virgolette{parallelo} i valori ottenuti sono
\begin{equation}\label{mq_parall}
 m/e = 5.938E-12 \pm 5.17E-13 \; \; q= -5.80E-10 \pm 2.53E-10
\end{equation}
Si osserva che in questo caso la compatibilit� con l'origine dell'intercetta ottenuta � al di l� del limite di confidenza del $95\%$ seppur di poco. Le figure \ref{retta1} e \ref{retta2} riportano una rappresentazione grafica di quanto ricavato sperimentralmente dai dati.

\begin{figure}[htpb]
	\centering
	\includegraphics[scale=1.25]{retta1}			%modificare percorso
	\caption{Regressione Lineare - perpendicolare.}\label{retta1}
\end{figure}	

\begin{figure}[htpb]
	\centering
	\includegraphics[scale=1.25]{retta2}			%modificare percorso
	\caption{Regressione Lineare - antiparallelo.}\label{retta1}
\end{figure}	

Con i dati ricavati da \ref{mq_perp} e \ref{mq_parall} � possibile invertire per ottenere i valori di $e/m$ calcolando l'errore tramite la derivata, in questo caso in una sola variabile. I valori ottenuti sono rispettivamente
\begin{equation}\label{em_perp}
 e/m = \SI{1.874E+11}{\coulomb/\kilogram} \pm 1.55E+10
\end{equation}
e 
\begin{equation}\label{em_perp}
 e/m = \SI{1.684E+11}{\coulomb/\kilogram}\pm 1.47E+10
\end{equation}
che risultano compatibili tra di loro e compatibli entrambi con il valore dichiarato di $\SI{1.7588E+11}{\coulomb/\kilogram}$.

\section{Conclusioni}
L'esperienza di verifica del rapporto $e/m$ pu� dirsi eseguita con successo. Entrambi i valori ottenuti, seguendo la stessa procedura ma con condizioni differenti, restituiscono infatti un valore compatibile con quello dichiarato. L'unico punto da chiarire resta la scarsa compatibilit� dell'intercetta con l'origine per la seconda regressione, sebbene il limite del $95\%$ non sia poi cos� distante. 
\end{document}


